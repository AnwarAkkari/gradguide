\documentclass[12pt]{article}
\usepackage[utf8]{inputenc}
\usepackage[T1]{fontenc}
\usepackage[letterpaper,margin=1in]{geometry}

\usepackage{graphicx}
\usepackage{libertinus}
\usepackage[kerning=true, tracking=true, spacing=true]{microtype}
\usepackage{setspace} 

\usepackage{etoolbox}
\AtBeginEnvironment{quote}{\singlespacing\small}
\usepackage{hyperref}
\usepackage{siunitx}
\usepackage[usenames, dvipsnames]{xcolor}

\newcommand\ytl[2]{
\parbox[b]{8em}{\hfill{\color{Maroon}\bfseries\rmfamily #1}~$\cdots\cdots$~}\makebox[0pt][c]{$\bullet$}\vrule\quad \parbox[c]{4.5cm}{\vspace{7pt}\color{MidnightBlue}\raggedright\rmfamily #2.\\[7pt]}\\[-3pt]}


\usepackage{xurl}

\begin{document}

\title{\vspace{-2em}Applying to STEM Ph.D. Programs}
\author{George Iskander}
\date{}
\maketitle
\vspace*{-2em}


\textbf{Grad Folder}: \href{https://drive.google.com/drive/folders/1vI2JCuJCNmggk-SFkw5LnA046zGItzvM?usp=sharing}{(Click here!)}

\textbf{Email:} \href{mailto:george.iskander@yale.edu}{george.iskander@yale.edu}

\textbf{If you want to see my CV and SOP, feel free to email me.}
\tableofcontents
\newpage


\section{What is this guide?}
This is a guide on how to apply to graduate school and fellowships. My philosophy in this guide is to give you the tools and knowledge you need to create a strong app. In the end, I will link to some great guides you may have heard of (Prof.\ Guo's for example). But my focus is different. Guo, for instance, emphasizes the role of research in the evaluation of your SOP and tells you what's important, but he does not tell you \textit{how} to craft a strong app. I hope I can give some pointers!

Caveat: I write for my experience in physics, and this advice will largely transfer to other STEM fields, with some exceptions. In mathematics, undergraduate research is difficult to conduct, and less emphasis is placed on the research a student has conducted, for instance.

\section{About Me (and why this guide?)}
My name is George, and I am a current senior. I will be entering a PhD program in physics this fall. I'm passionate about physics as well as about outreach in STEM. Along my journey, I've been helped by amazing mentors, and the only reason I am where I am is because of them and their generosity. I wrote this guide because, in my opinion, graduate admissions is much more opaque compared to undergraduate admissions, and I believe this guide might be helpful to current juniors thinking about applying.

For instance, there's typically many factors that are beyond your control in graduate admissions. Strong candidates get rejected all the time, even if they were a good fit for the program and the faculty. Why? Maybe a person didn't get in because all the faculty they want to work with aren't taking students. Or because they didn't talk to faculty, and some other student did talk to the faculty. And this is what makes the process opaque. People will tell you to talk to faculty, make connections, sure, but I find this advice sort of vague.

So, my goal with this document is to cut through the cruft and distill the most important advice I've gotten. I want you to know what the process is like and how to optimize your chances. I know that the idea of having a non-insignificant part of your career in someone else's hands is really frustrating, especially when you know you're being judged by metrics that you may find unfair (such as the GRE and GRE subject tests). So, I spent months thinking about how to improve my application. In retrospect, I realize there is only so much you can do, but there are some important things you can do. I got rejected from a lot of schools, but I got into many other programs that I'm excited about! 

At the end of the day, it is a crapshoot. You will get in somewhere, absolutely. The question is just where. I hope with this guide in hand, you are more informed about the process and can craft the absolute strongest application you can. At the top of this guide, you will find a link to a Google Drive folder that has a lot of resources as well as a Google Sheet that will help you track your app (I call it simply the ``grad sheet'' in this document).

I've also put my email at the beginning of the document. Do not hesitate to email me if you have any questions and need any help. I want to make admissions equitable, and I believe the best way to do so is to make myself as available as possible.

\section{Tips for Freshmen/Sophomores}
\subsection{Comments}
If you find this document and you're a freshman or sophomore, you might be considering going to grad school, and you're worried about what to do now to ensure you have a strong application later. This is definitely something to think about and take action on, but it is not something to stress about. Getting into grad school is predicated on being a good student and doing well in research and to a certain extent, your classes! So here's the advice I will give you:
\section{Advice}
\begin{enumerate}
\item Do well in your classes and go to office hours. At the beginning, I mentioned GPA is not the most important part of the application. This is still true here, but all things equal, you want to ensure you do the best you can in your major classes. If you've had a bad semester, it's absolutely not the end. I had a very tough time in my first three semesters, and demonstrated an upward trend after that in my application! Showing an upward trend demonstrates to the admissions committee your tenacity and your ability to do well.

Go to office hours as well, for any reason: if you have questions about the material, think the professors is cool, or you just want to make a connection. Once it comes time to ask for recommendation letters, having this connection with your recommenders will be very beneficial.

\item Get involved in research, whether on campus or at external programs. I could write another document just on this topic, but I've written some FAQs on this subject so I'll link them here:
\begin{enumerate}
\item \href{https://spsyale.sites.yale.edu/research-guidefaq}{Research Guides/FAQ (click here)}
\item \href{https://spsyale.sites.yale.edu/resources}{List of Research Programs (click here)}
\end{enumerate}
I must really emphasize this point. Research is important, and having letters of recommendation from research advisors can be a very strong boost.

\item If you have any questions, I can't say it enough: email me anytime! I'm more than happy to help and answer any questions out there.
\end{enumerate}

\section{The Question of GPA}
I need to dispel this from the outset. I've had a lot of people ask me if they can get into grad school with anything other than a 4.0. I want to say that when it comes to reading your application, GPA is far from the most important thing in your application. Grad schools care first and foremost about your ability to be a capable researcher. GPA does not reflect your ability to do research. That is not to say it's not important. I will say that once you're over a certain GPA, it doesn't really matter what the exact number is. This number depends on the school, but once you're past this hurdle, your GPA is not going to be the reason you've been rejected (if you get rejected). So, if you are set on going to graduate school, give it a shot, and don't worry about your GPA.

\section{M.O. of Graduate Admissions}
How do grad committees judge your app? They are looking to answer this one question: \textbf{do you have the ability to conduct graduate-level research and can you be a productive researcher at our school with one of our faculty members?} \footnote{There is a caveat. Say the professor you want to work with decided to take a sabbatical or to stop taking students. So, the answer to this question might be yes, but you may not be accepted, even though there's no doubt you are a strong candidate. \textit{C'est la vie.}} You should have this question in mind when drafting your materials. Everything you prepare should address this question in some capacity. I don't want to scare you or sound didactic, but fully-funded Ph.D. programs will be giving you a tuition waiver and paying you a decent stipend for 5-6 years. This is a commitment of a few hundred thousand dollars. This is why they ask this question, and why you must answer it with a firm ``YES.''

This is why fit is important. An arbitrary example: you can be the best biologist in the entire world, but if I'm studying history, there's no way we're going to be a good fit for one another and carry out successful research. Fit is something you need to be aware of. If no faculty member is asking the same questions as you, you won't be a good fit.


\subsection{How the Adcom Works}
The adcom is the admissions committee. What happens is there's a set of professors from the department who are on the committee. Their job is to read applications. If you mention that you want to work with a specific faculty in your SOP, your application might get forwarded to that specific faculty member, and they will be asked to give some feedback on your application.

More or less, this is how it works. Some schools do employ cuts and will not look at some apps with test scores below some threshold. I think the number of schools who cut is fairly low though. 

The important thing about admissions is that your app is being read by professors in the actual department, rather than administrators. I will discuss later why this means you should contact faculty.

\section{Testing}
\subsection{Comments}
Depending on your field, you may have to take standardized tests, typically the GRE and subject GRE. This is the part of this document that may be most out-of-date when you read it; many programs are phasing out the GRE and subject GRE, but this is highly dependent on field and school. But, it is very likely you must take the GRE, and depending, the subject GRE. I would strongly recommend taking the subject GRE if even only a couple of schools you're interested in require it, since you do not want to be in a situation where you want to apply but you can't take/haven't taken the test. Here is my advice.
\subsection{General GRE Advice}
\begin{enumerate}
\item In contrast to the subject test, the general GRE is offered quite literally every day of the year. You have a lot of freedom in choosing when to take it. I recommend taking it no later than early October. You want to give yourself some time to take it again. ETS stipulates that you can take the GRE once every 21 days.

\item The GRE is pretty similar to the SAT, with some important differences. It's computer-delivered and has two essays. There's two math sections and two verbal sections with an additional experimental section (which could either be math or verbal). 

\item I recommend the ETS study guide, it has a lot of helpful tips for doing well, as well as a ton of practice questions. The types of questions you might be asked are pretty different from those you've encountered on the SAT or ACT, so you'll want to be familiar.

\item At the testing center, you're allowed one break for the exam. Take a snack for the break as well as a bottle of water. Get rest the night before, and show up in your comfiest clothes since you will be there for a couple of hours.

\item Don't stress too much about the GRE. It's not as important as other parts of your application.

\item If you want resources for the GRE, email me and I can send them.

\end{enumerate}


\subsection{Subject GRE Advice}
\begin{enumerate}
\item Start prepping early, and use the summer wisely, since this is when you have the most time to study. The GRE is offered every day, and the subject tests are offered three times a year: April, September, and October.

\item If you have the time, it doesn't hurt to take the test in April: if you do well, you're done. And if you don't do as well as you want, you know what the test is like and can better prepare for the September and October tests.

\item If you are taking the test in the fall, I recommend signing up for both the September and October administrations. Why? By the time September results are released, it's too late to sign up for the October test date. So, if you didn't do as well as you wanted to on the September test, you're out of options. For this reason, I recommend signing up for both. If you do well in September, you can cancel the October administration and you will receive 50\% back.

\item If you need any links to resources, email me and I can point you to individual resources for the different subject tests (books, released practice tests, and so on).

\item Don't focus on your score too much. I know plenty of people who got into top-tier programs with scores they thought were not good enough. Do the best you can, but realize, at the end of the day, this is a small factor. What matters more is your experience, SOP, and letters.
\end{enumerate}

\section{Letters of Recommendation (LORs)}
\subsection{Comments}
So, let's start with one of the most fundamental parts of the application, which is common to every application you will submit. LORS: you need them. You're undoubtedly familiar with them due to undergrad and summer jobs/research. For the most part, LORs are just a formality in undergrad admissions. But, in grad admissions, they make all the difference. Let me dive in.

\subsection{Advice}
\begin{enumerate}
\item \textbf{In grad admissions, letters are one of the most important aspects of your application.} I've heard from faculty and administrators that a good letter can erase almost any blemish on your application. This is why securing good ones is important.

\item Typically, you need three. Some apps I've seen ask for two. You will never need more than three, but some apps will let you add an optional fourth. It definitely doesn't hurt to ask for one, but only if you believe it would be strong.

\item What makes a strong letter? Remember the M.O. The best letters will attest to your ability to carry out research. So, ideally, you should get one from every person you've researched under. A letter from your research advisor answers definitely that you can do research. So-called ``research letters'' are crucial.

\item How do you ensure these letters are as strong as can be? The advice I've gotten is to ask ``Do you think you can write me a strong letter of recommendation?'' 

I think a good way to approach it is to schedule a conversation with the people you intend to ask to talk about your ``future plans.'' When you meet, talk about the fact you're applying, perhaps what schools you're applying to, and at the end, ask if they can write you a ``strong'' letter of recommendation.

\item When should you schedule talking to recommenders? I recommend perhaps in the last few weeks of the spring semester. This way, you give them a heads up, and before the hectic fall semester.

\item When you ask for a letter, you should also offer to forward your CV and transcript to help them write their letter, as well.

\item One very important thing: if you believe an aspect of your application is weak, ask your recommender to address it in their letter. This is actually one of the best ways you can address a weakness, since it sounds less like an excuse coming from the recommender. A weakness can be something like poor GPA, poor standardized test scores, things like that.

\item Keep your recommenders in the loop. When the time comes, send them a draft of your SOP. This will help them get a feel for your specific interests, and they can tailor their letter better. The more involved they are and the more in-the-loop they are, the better. 

\item Track the status of the letters in the grad sheet. If you apply to ten programs and ask for three letters, that's thirty letters to keep track of. So, use the sheet to make sure everything gets in on time. Remind your recommenders every so often through the semester, and give them a big heads up about two weeks before the application deadline.
\end{enumerate}


\section{Finding Programs}
\subsection{Comments}
Like I've said a couple of times, the application process can be opaque. One reason is that now, you gotta do more homework about the places you're applying to than you probably did for undergrad. You probably picked your undergrad institution based on some metrics like distance from home and prestige, but the primary rationale behind your choice of school most likely came down to subjective factors like feel, social environment, vibe, and so on. You go with wherever makes you feel good, since you're largely getting the same education across the board.

Grad school is a different beast. There's still some simple metrics like prestige and reputation, but now you suddenly have a bazillion more factors to consider: the faculty there, the work-life balance in their group, average time to completion, the support from the graduate cohort, and so on. It's often-times not as clear. So, let's discuss how to make a list of programs.

\subsection{Advice}
\begin{enumerate}
	\item I'll give you the candid advice one grad student at Caltech once told me: ``Grad school is tough, and I'd rather be depressed in California than New York.'' I'd have said it a bit differently, but there is a deep kernel of truth there, and it's that location matters a whole, whole lot. Maybe location matters to you because you like the sun, you want to be close to your SO/family, you don't like cities --- everyone has a preference, and you absolutely need to take this into account.

	I can't tell you how many students and professors have given me this advice. For whatever you want to do, there's probably 50 amazing graduate institutions doing that sort of thing. You obviously know yourself and know that you would be happier in one place than another. So don't apply to a place that you think you would absolutely hate. What's the point? You can afford to be picky. 

	\item Start your search by first looking for people and research, not schools. I can tell you how I came into my research interest and schools, and maybe this will help. I'm interested in low-energy and non-accelerator tests of fundamental physics. The field is well recognized, but still nascent. I was telling a friend about how cool I think this field is and she told me to look at a conference that was exactly on this field of physics. I looked through the conference proceedings and I saw a lot of amazing projects I was interested in, and I took note of who was presenting and from where. I wrote the names down. And already I had a preliminary list of schools and faculty. I kept looking into the literature and I found more faculty and more programs this way.

	Doing this helps because it avoids the trap of prestige-chasing. Everybody wants to go to Harvard, MIT, and Stanford, sure. But maybe there's nobody there who actually fits what you want to do. And there's maybe schools that are not as well-known that are doing exactly what you like. Looking by people and research instead of by school frees you from your biases.

	\item When you write down the names of these people, write a short little note on what they do. If you're like me, you're going to totally forget what everybody does. If you apply to ten school and each has three interesting faculty members, that's thirty people with different projects. You will forget what they do from time-to-time, and you don't want to dive down the rabbit-hole more than you have to. 

	I guarantee you will learn so much about all the people in the field through this process, and in a few months, you will feel so cool when people name a person and you're like, ``Oh I know them!''

	\item Talk to your advisors. Your advisors/research advisor(s) know the field like the back of their hand, so they can definitely point you in the way of people they think might be a good fit. I think I came into a lot of interesting people I never knew about just through conversations with faculty. 

	That being said, don't be afraid to hit up faculty at your own school if you think they might help, even if you don't know them. In my experience, faculty are happy to help students who are applying. Moral: Talk to people.

	\item There will inevitably be a ton of cool programs, but don't apply to all of them. I applied to 15, and I could have made that 20+, but I didn't. For multiple reasons: money, location wasn't that great, and so on. As the process becomes more and more selective each year, it becomes more common for people to apply to more. But I'd cap it at 15. Preparing all those applications becomes a lot of effort. Doing 20 apps well is a lot harder than doing 10 well. So, be wise, and don't go off the deep end.

\end{enumerate}

\section{Talking to Faculty}

\subsection{Comments}
The app process has a strong ``Who do you know?'' component to it. What I mean is this: I've talked about fit, right? Professors care that you'd fit in their research group. Remember the M.O. They're making a pretty big commitment if they admit you to the program, so they want to make sure you'd have a good time with the faculty there. This is where talking to faculty comes in. If, before the app process starts, you reach out and talk to faculty, you can establish a connection and tell them exactly how you'd fit in their research group. 

Generally, professors have some type of pull on who the adcom admits. So, make a good impression and you can improve your chances. The amount of pull professors have depends on the school. So caveat: it's not a guarantee.

Once you've found your list of programs and faculty, then you can start reaching out.

\subsection{Advice}
\begin{enumerate}
\item When is the best time to reach out? The advice I've gotten is that the early fall is best, but honestly, if you're going to be a senior in the fall, then you are going to be pretty busy (classes, senior project/thesis, the emotions of being a senior, as well as overall stress of applications), so I recommend contacting faculty sometime between mid- and late-summer, say from mid-June onwards. Faculty will be less busy in the summer since they're most likely not teaching then.

\item So what do you say? This is where your research comes in. Introduce yourself, mention what you want to do, describe what about their research is interesting, then ask if you can chat at some point. Overall, keep it succinct. The more succinct, the better the chance you'll hear a response back.

\item Most of the time though, you'll have to follow up. Wait a week before following up. Also use GMail's schedule send feature; it's very useful for sending emails at a particular time of day.

\item If they respond and agree to chat, what do you talk about? There's a lot of questions to ask. First, ask whatever it is you're curious about. I believe these conversations work best if you have questions you genuinely do want to ask. But, there are questions you should be thinking about that are important:
\begin{enumerate}
\item Is that professor taking students next year?
\item How collaborative is the group?
\item What is the average time to completion?
\item What do they think their research program will look like in the next few years?
\end{enumerate}

\item Be sure to thank them for their time. And stay in touch from time-to-time. I stayed in touch with faculty by asking how their research was going and giving updates on my own research and classes. Doesn't have to be long.
\end{enumerate}

\subsection{Email Template}

Here's a template you can use to email faculty. Don't feel like you need to use these exact words, you may have to change it to fit whatever it is you want to say.
\begin{quote}
Dear Prof. XX,

I hope this email finds you well. My name is \rule{1cm}{0.15mm} and I'm a rising senior in the \rule{1cm}{0.15mm} major at \rule{1cm}{0.15mm}. I've done research in \rule{1cm}{0.15mm}. I am looking to apply my skills from these experiences to \rule{1cm}{0.15mm}. I will be applying to \rule{1cm}{0.15mm}'s PhD program in \rule{1cm}{0.15mm} this fall, and I would be interested in joining your group, if there are any openings.

I recently read your papers on \rule{1cm}{0.15mm}, and I was curious about \rule{3cm}{0.15mm}.

I've attached my CV below for your consideration, and I'd be very interested to talk more about your work and my interests if you have a chance. I look forward to hearing from you soon.

Regards,

\end{quote}

\section{Fellowships}
\subsection{Comments}
So, what are fellowships? They're basically like  (with some important differences) scholarships for graduate school. If you're applying to a fully-funded PhD program, why even apply, you might be thinking. Good question. A fellowship will maybe just give you a larger stipend, but it may not be substantially more. The reason why you should absolutely apply to fellowships is three-fold: opportunities, freedom, and community. Let's go over these.

Caveat: Many are reserved specifically for U.S. citizens or permanent residents (typically the federal ones). But many private ones usually don't have any citizenship restrictions, so it's worth checking those out especially.

\textsc{Opportunities.} A lot of fellowships have instant name-recognition, and having their support will further your career. Schools like to see people who can bring in money, so when you apply to postdoc positions (if you choose to do so), then having a fellowship can certainly help in being hired. Some fellowships have the goal of recruiting students to a particular line of work. Take for example, the DOE NNSA SSGF, which tries to heavily recruit students to national labs. If you do this fellowship, you are required to intern at a government lab for a summer. From what I've heard from current fellows, if you win the SSGF, getting a job at a national lab becomes slight work, basically. So, it opens doors to you \textit{just} by virtue of having it.

\textsc{Freedom.} Sometimes, a PoI (person of interest) at your school may not have space to take you on in their group because of funding. This is a serious problem you might face. However, if you bring in your money via fellowship, the game changes, and you can essentially work with anyone barring other factors (perhaps the faculty is too busy with their other students; unfortunately, money can't fix that). Disregarding those other factors, you are unrestricted in the choice of advisor given that you win a fellowship.

\textsc{Community.} A lot of fellowships, especially those dedicated to increasing the number of underrepresented students in academia, have annual conferences you are required to attend, where you can meet faculty, peers, and administrators. I've heard from people that these conferences are often the highlight of their fellowship experience. You get to share your struggles, your research, and you expand your network. And presenting your research always helps further your career too.

I think a large percentage of students will hold off on applying to fellowships until they're in graduate school. I strongly, strongly recommend you apply as an undergrad, and there's many reasons why I recommend this. Primarily, most fellowships allow you to apply multiple times, usually until your second year. If you apply now, you get experience with applying that will help even if you don't get it. If you get one as an undergrad, you are freed from TA responsibilities in your first year, and can spend more time getting acquainted with research.

Now, we'll go over some general advice on applying to fellowships.

\subsection{Advice}
\begin{enumerate}
\item Apply early. All the deadlines change each year, but many of the fellowships are due as early as mid- to late-October. So you’ll want to start drafting your statements over the summer, and secure your letters as soon as you can.

\item Don’t be afraid to reuse statements. 

\item Many of the fellowships ask for a personal statement describing your journey, what led you to your field of study, adversity you’ve overcome, and so on. It is crucial you emphasize if you are underrepresented in these statements. This is actually one of the criteria for the NSF GRFP, and it goes without saying that this is important for the fellowships for underrepresented students. These fellowships define underrepresented (URM from here on out) as first-generation, low-income, racial minority, disabled, sexual or gender minority, and so on. Do check how each fellowship defines URM.

\item Most ask for a research proposal describing research you want to do. Work on these early and polish them as much as you can. Clear writing is paramount. I recommend brainstorming ideas and then discussing them with your research mentor. Do not be afraid to lean on your mentors. 

\item You \textit{must} make clear how the fellowship will support you. Be extremely explicit by saying, ``With the support of X fellowship, I will do ---'' then hit them with a punchy line about the exact thing you will do. This is crucial. Describe what specifically about the fellowship will help you. The NSF offers free supercomputing time, for example. If this will help your research, say: ``With the supercomputing time offered by the GRFP, I will ---'' and so on. 

\item Look at the criteria each fellowship uses. Let’s use one important example. The NSF judges GRFP applications based on two sets of criteria. The first is called Broader Impacts (BI), the second is called Intellectual Merit (IM). IM is just what it sounds like: does it make sense in the context of the literature? Are you answering a question in your research proposal? Is your research meaningful? Now, the BI decides who wins. The BI asks how involved you are with URM communities, how your research serves to connect academia and industry, how your work helps the United States, etc. You don’t need to address every one of the over ten BI criteria, but it is important that you tailor your application to address what you can. You must pay attention to this. At the end of the day, the NSF sits down and literally grades how well you address the BI. Every other fellowship is doing the same with their own sets of criteria, and they’re always available on their websites. Look at the criteria, think about them, tailor your app to them, and this will put you ahead.

\item Use fellowships as leverage. They literally change the game. I’ve heard of schools that browse the list of GRFP winners and extend them offers solely because they’ve gotten the GRFP. It’s nuts. Winning one of these fellowships makes the school’s decision a lot easier. Normally, they’re deciding if they want to make a 5-year, \$250,000 commitment to you. With a fellowship, they have to pay less for you. If you win a fellowship or advance in a round, inform your schools ASAP. I have also heard of people getting rejected from schools and then later getting a fellowship. They then nicely call up their schools and say, “Hey, I won a fellowship?” And boom, they change that rejection to an acceptance just like that. Caveat: it’s still not a guarantee, and schools are moving to disregarding funding when making decisions. But still: apply, since the risk is low, and the benefits of winning are huge. And even if you don’t win, you know how to strengthen your app for next year.

\item  It is in your interest to apply to multiple, since you can usually defer them so that one supports you once the other ends. Note, if you win multiple federal fellowships (DOE, DOD, NSF, what have you), you are only allowed to take one federal fellowship. But there’s no restriction between a federal+private fellowships or private+private fellowships. If you win multiple and decline one, put it on your CV anyway, saying ``(declined)'' after the fellowship name. This will be important once you start applying to other academic positions, because schools want to see you can bring money in.

\item Secure 3+ letters, if possible. When it comes to grad schools, most of them just want three letters. Many fellowships however, ask for 3 minimum and 5 maximum, with 3 being the typical number applicants submit. If you can procure a strong fourth letter (same with fifth), then it is in your interest to submit that as well. Having multiple strong letters helps when it comes to these fellowships. But don’t stress about this. Most applicants win them with three letters. And don’t add extra letters to have extra letters. Only add those that are strong.


\item Even if you don't think you can win, you should try. You have nothing to lose and a lot to gain. At best, of course, you win. If you don't, you can still get honorable mention. Getting honorable mention helps a lot if you choose to re-apply the following year. 

\item That brings me to re-applying. Don't be afraid to re-apply. Even if you don't win, you'll be armed with a lot of information you can use to strengthen your app the following year.
\end{enumerate}
\subsection{List of Fellowships}

\subsubsection{Ford Foundation Predoctoral Fellowship}

This fellowship is for underrepresented students who are planning to stay in academia. Ford supports three year of graduate study, and they also host an annual conference you are invited to if you are a recipient of the fellowship. I have heard incredible things about the fellowship and the cohort, and the fellowship offers additional fellowships to winners of the predoctoral fellowship.

\subsubsection{NSF GRFP}

One of the most prestigious graduate fellowships out there. The NSF will support three years of graduate study, and benefits include: free supercomputing time and the ability to apply for funding for domestic and foreign internships. You can apply as many times as you want before graduate school, but once in graduate school, you will only be able to apply once.

Super, super important to look at the BI for this. Read the solicitation once it drops in September, it’s a document that details how many awards the NSF is giving out, how much money, the criteria for that year’s fellowship, and so on. The solicitation is your Bible. That one document has EVERYTHING you need to know about that year’s GRFP.

\subsubsection{DOD NDSEG}

This fellowship pays for three years. Your research does not have to have military applications to become funded, but you should have a strong focus on practical applications in your research statement. The DOD also reimburses travel, which is nice. Do start this one early, since the research statement is four pages (three pages plus one page of references). It is similarly prestigious to the NSF GRFP, and pays a bit more, and reimburses travel as well.

\subsubsection{DOE NNSA SSGF}

The SSGF pays for four years. This one is a bit niche: the SSGF supports research in high density physics, nuclear physics, and hydrodynamics, all to the end of improving U.S. stockpile stewardship. Again, your work does not have to have direct applications to stockpile stewardship, but you should spend some time connecting your work to stockpile stewardship, since this is what separates the winners from the non-winners. 

The SSGF tries to recruit students to national labs, and of course, you aren't bound to work in one after  graduation (though the SSGF stipulates you need to work at a national lab for at least one summer). But, it certainly gives you a big foot in the door for post-PhD opportunities.

\subsubsection{DOE CSGF}

This fellowship pays for four years, and is tailored to those doing computational research. The application is very similar to the SSGF as well: you have to connect your work to the energy goals of the DOE. This is a more well-known fellowship, and carries similar prestige to the NDSEG and GRFP.

\subsubsection{Stanford Knight-Hennessey}

This fellowship will support your study for three years at Stanford, and your associated department will foot the rest of your salary/tuition for the rest of your time there. Very competitive, but incredibly prestigious. The program pays for a stipend and covers the cost of living in a brand-new Stanford dorm. They have a very unique and light-hearted application which is not too strenuous to fill out.

\subsubsection{Hertz Foundation Fellowship}

The ultimate \textit{crème de la crème} of fellowships. The Hertz is incredibly competitive, but will pay for five years of support. Among its alumni are two Nobel Prize Laureates and the founder of Google X. Definitely worth a shot, but it is very hard to get it due to the competition and nature of the fellowship. I have read it includes two difficulty interview components. Far be it from me to dissuade you though; I firmly believe in casting a wide net rather a small one.

\subsubsection{NPSC Fellowship (Now GFSD Fellowship)
} 

A lesser known fellowship, but it will support you for up to six years, albeit with a lesser stipend. But having six years of support is unheard of, and the fellowship will also hook you up with industry internships. I talked to one alumnus who said the fellowship was crucial to helping them in their career path to NIST. The fellowship application is simpler than most others, and is open to anyone in STEM.

\subsubsection{P.D. Soros Fellowship for New Americans}

The P.D. Soros supports up to two years of study. The Soros will fund any postgraduate program you wish to do: PhD, MD, JD, MBA, and so on. The P.D. Soros is specifically for immigrants, children of immigrants, refugees, asylees, and DACA recipients. They’re always making the eligibility criteria as inclusive as possible, so consult with their website for the final word on if you’re eligible.

The Soros is pretty prestigious, and is open to undergraduates and people in the first two years of their graduate study. However, it is becoming more and more selective, so don’t be dismayed if you don’t get it the first time. This year (2020), the acceptance rate is about 1.5\%.

\subsubsection{MUREP Aeronautics Scholarship and Advanced STEM Training and Research Fellowship
}

The website is a bit cryptic about how long the fellowship lasts, but it is funded by NASA, so it’s legit. Frankly speaking as well, it is lesser-known, so I assume it is less competitive and thus, easier to win. The fellowship aims to support underrepresented students in STEM.

\subsubsection{GEM Fellowship}

A fellowship for underrepresented students. The nice thing about this fellowship is they match you with a summer industry internship before graduate school, giving you phenomenal industry connections at places like Intel, Apple, etc. before you go and start your PhD. The only downside is this internship has to be the summer before you start graduate school. If you don’t mind this, definitely apply. The deadline is a bit earlier than other schools (mid-October!)

\subsubsection{DOE NNSA LRGF}

This fellowship is a close sister to the SSGF. It has a similar goal of placing students at national labs, but it's not hyper-focused on nuclear stockpile stewardship. This fellowship is very new, so the DOE is soliciting as many applications as they can. The caveat is you have to be a current grad student to apply. But, I've only heard wonderful things about it, and like the SSGF, this gives you a huge legup in applying to postdocs and faculty positions at national laboratories.

\subsubsection{Google PhD Fellowship Program}

Google maintains their own fellowship mainly geared towards different areas in CS (machine learning, systems and networking, algorithms) and quantum computing. And the name speaks for itself, Google will treat you right if you win this one. They will pay for up to three years of study, give you cloud computing resources, and they will match you to a research mentor in their research division.

\subsubsection{NSF QISE-NET Fellowship}

I'll be upfront and say the format of this fellowship is a little mysterious. Anyways, I'll do my best. So this is a fellowship sponsored by the NSF to bridge together academia and industry. Applications are accepted from ``triplets,'' a.k.a. you + your PI + another PI from industry (either private industry or a national laboratory) submit a proposal together on some quantum information or quantum engineering project. If accepted, all three of you work on the project for a couple of years. It's very new, but has already had a diverse and expansive cohort working at places like Google, IBM, and IonQ.

\section{Diversity Statement}
\subsection{Comments}
So, the diversity statement is an optional and additional statement for many schools. It differs a lot between schools - some will only let you type up to 150 words, others will ask for a page or two. In any case, it’s almost always optional, and usually doesn’t affect the actual admission. It will only really affect whether you’re considered for a diversity fellowship or not. They will only review the diversity statement once they’ve decided to take you. So, it’s in your best interest to include it. But what is diversity? What makes a compelling diversity statement?

\subsection{Advice}
\begin{enumerate}
\item A preface: I’d start with typing a single diversity statement that’s one to two pages, then modify and cut for each school. I think that’s easier than writing from scratch for each. If you’re applying to fellowships, you can usually reuse your statement, so the work is cut out for you.


\item Now, the meat: what’s good to write about? I think really almost anything can be valuable to discuss in the context of diversity. What’s crucial is discussing how you plan to give to the school with your efforts or what you’ve learned from your experience. An example from my own experience: I’m first-gen, and I struggled a lot in college, so I got involved with mentorship to help students. I can keep being a mentor to undergraduates in grad school in order to build a more inclusive department and academic environment. In this narrative I was able to discuss my own background as a first-gen student, and how I was able to use that to make my school a bit more inclusive. 

\item And you can do this with whatever your experience is: a health struggle, academic struggles, struggles with identity, and so on. Just try to think about your experience in undergrad that’s been impacted by your status as a minority/URM, and how you plan to bring what you’ve learned to the school.

\item  What if you’re not an URM? Not an issue. I think a good angle for you to take is to discuss how you recognize you’re not URM but are invested in helping URM and diverse students. However, the bar is higher. You’ll need to put your money where your mouth is and back this up with action. If you have volunteer experience, if you’ve given back to the community, point to this as concrete evidence that you’re invested in making your field of study more inclusive. 
\end{enumerate}

\section{Fee Waivers}
\subsection{Comments}
The process is expensive. Each university charges about \$100 to apply. If you apply even to just five programs (and consider the average for a lot of STEM programs is ten and above), you will pay a lot of money. So it’s in your best interest to secure the fee waivers, if you qualify.

\subsection{Advice}
\begin{enumerate}
\item Get the fee waivers \textbf{as soon as possible}. Some schools will actually have them first-come, first-served. Do it early.

\item Now, the way you get them: Each school has an FAQ which describes eligibility requirements. It could be your EFC is 0, you participated in a diversity program, you had some amount of financial aid at your undergrad, and so on. What’s nice is most schools just have you check a box on the app saying you’re eligible, and you’re done. Some are a bit more involved and will ask that you email them with proof of eligibility to get the fee waiver. Considering you have to wait for them to get back to you, again, it’s best you do it early.

\item You can also get fee waivers at conferences! A lot of minority-oriented conferences will have recruiters who will give out fee waiver codes, and this saves you the time it would take to apply for one. If you're attending a conference, it's good to meet recruiters and get the fee waivers then.

\item I recommend just Googling each school/department’s policy, and keeping it all in the grad sheet, so you can keep track of it all.
\end{enumerate}

\section{The Application}
\subsection{Comments}

So, the app usually opens in August/September, and it is usually due anywhere from mid-December to mid-January. So, what is it like?

\subsection{Advice}
\begin{enumerate}
\item It’s basically just like undergrad. Biographical info, citizenship, background info, and so on.

\item They’ll also usually ask for your CV + SOP. With the letters of rec, this is basically it.

\item Some apps will ask you to list: every class you’ve taken in your major, your grade in all those classes, what year you took them, and what textbook you used. This can be exhausting to fill out, so I recommend you take a look at the class tab of the grad sheet, and just spend some time filling it out so you can easily copy from it.

\end{enumerate}

\section{Statement of Purpose (SOP)}

\subsection{Comments}
Ah, now we get to the real heart of the app. If you are a senior, most of what's really left in your control is the SOP. The advice I got about my application was ``knock it out of the park.'' Kind of stressful when you are already stressed. But I'm going to tell you how to create a strong one and how to crush it.

\subsection{Advice}
\begin{enumerate}
	\item Before you even begin, I want you to make a list of everyone you believe wouldn't mind giving it a look. I aimed to have it in such a state that by the time I sent my SOP for review to the last few people on my list, they would have no comments or suggestions. This ensured that I really did have something strong, since different people would all agree it's strong. Now, I think you should try to come up with as many people as you can. Younger faculty are great, since they know how it was not too long ago. And current graduate students will know exactly how to help. 

	Don't be afraid to write down people whom you haven't really been in touch with. I've found people are really willing to help in the grad admissions process if you're polite and give them time. Moral: just write down \textit{anybody} you think would be happy to help.

	\item Also pay attention to anything at your school that may help. My school hosted some SOP review workshops with current grad students, for example. Going to something like this can really help you get an important perspective. So, just be aware of what's going on and what may be helpful.

	\item Now we come to what you should write. I have a lot to say about this. The statement of purpose ultimately needs to be a narrative, but one that is professional and succinct. I'm not talking like your common app. I'm talking like a narrative that covers what research you've done in undergrad,  what your research interests are, why that program, and what you can bring. Remember the M.O. It's a lot to cover, which is why most (not all) schools usually will not impose a limit on how much you can write. Keep it to two pages maximum though, they will start skimming if it's any longer.

	But my point is that it should be cohesive. So, in the next few points, I'll discuss what needs to be covered and how to do so.

	\item A warning: there are some \textbf{kisses of death} in the SOP. These will basically kill your application. \href{https://psychology.unl.edu/psichi/Graduate_School_Application_Kisses_of_Death.pdf}{There is a study on this (click here!).} Two I think that are especially relevant: oversharing your struggles (especially with mental health) and trying too hard to impress. The first one is a rather unfortunate one; it is my personal belief that in an ideal world, students should be able to be entirely honest about their struggles with mental health. Unfortunately, academia has an unfortunate stigma, and sharing about mental health or other struggles will not help. So, it is best to omit any mentions of mental health. Do discuss your struggles, but do not focus on the negative; focus on your response and ability to overcome. Lastly, don't try to oversell yourself. I believe you're awesome, but we're all just getting started in our career paths, so be honest and confident, but don't go overboard.

	Additionally: no stories of childhood. I've seen apps that mention not to do this. Now, I will tell you again because it bears mentioning, \textbf{don't do it in any app!}

	Another kiss of death is discussing unrelated stuff from your CV in your application. If you ever read Philip Guo's guide, he emphasizes that when he reads SOPs, he is literally only looking to read about what research you've done and what research you're interested in. Your outreach, tutoring, is all rightly important, but you do \textit{not} want to include this in the SOP, because it does not address your capability to be a researcher. Remember the M.O.

	\item Your first draft will not be phenomenal. I wish it could be so, my first draft legitimately bad. I made a lot of mistakes and devoted a non-insignificant part of it to talking about my outreach. Save that for the fellowship essays, don't do it in the SOP. What you want to do is probably draft it, shelve it for a few days, and come back to it. And edit it, again and again and again, till you feel there's no more progress to be made on it. Then, start showing it to other people. I'd recommend sharing with graduate student peers first, since they're busy but will make time to read multiple drafts. Faculty usually will not have this patience, and you'll want to show them a draft once it's been a bit more refined.

	\item You'll want to begin the SOP with a strong and extremely concise introduction. I started by saying, here's what's going on the field, this is what I want to work on, and school XX is the place I want to do that because of their *insert program/unique features of program*. Three sentences, and you don't need more than that. 

	\item So, what you're probably going to do in the rest of the SOP is walk through your research experiences. I want you to describe a couple of things about each experience. Let me list.

	\begin{enumerate}
	\item What led you to that experience.
	\item What was the experience, briefly?
	\item The hardships and difficulties you overcame (with numbers if you can).
	\item What you've learned from that experience and how it will be applicable to your future study.
	\item (Optional) If your field of future study is different than what you've done, why.
	\end{enumerate}
	
	You may have many, many questions right now, as I did when I was in your shoes. What if you had experiences that weren't successful? What if you had an experience you didn't like?

	Here is one of the most important pieces of advice I'll give you: \textbf{spin it however you can, and find every positive you can.} If you had an experience you didn't enjoy, don't focus on what you didn't enjoy: focus on what skills you learned, how that experience helped you clarify your interests, etc. If your experience wasn't successful, did you learn how to change the way you approach problems? I had an experience which I thought would hurt me more than help me, but I wrote a lot about the really important skills I learned from that experience and how those would help me in my future research.

	\item I'll give an example from one of my statements, since this all sounds super abstract. We'll dissect it and see what we can learn from it.

\begin{quote}
	I spent the summer of 2018 at Caltech working with Prof.\ XX on FQNET, a quantum network based at Fermilab. FQNET is a platform for conducting quantum network R\&D and studying questions of fundamental physics, e.g., whether space-time is generated by quantum entanglement. At Caltech, we commissioned an entanglement source for the experiment. To this end, I designed and assembled a feedback controller for an electro-optic modulator in order to achieve high-fidelity entangled photon pairs. I found that my attempts to implement traditional PID feedback control failed. However, by programming our own nonlinear algorithm, we achieved stable feedback. Our \$50 controller performed comparably to a \$1000 commercial controller. We inferred a fidelity of 0.9938, which I reported in a first-author publication [1]. Toward the end of the summer, I used single photon detectors to show that we had realized entangled pairs of photons. During my summer at Caltech, I learned how to apply quantum information science to questions in fundamental physics.
\end{quote}

	Most of my reviewers remarked that this paragraph was one of the strongest in my statement. Now, sentence-by-sentence.

	\begin{enumerate}
	\item ``I spent the summer of 2018 at Caltech working with Prof.\ XX on FQNET, a quantum network based at Fermilab.'' Not much to say here, just a short intro.

	\item ``FQNET is a platform for conducting quantum network R\&D and studying questions of fundamental physics, e.g., whether space-time is generated by quantum entanglement. At Caltech, we commissioned an entanglement source for the experiment. To this end, I designed and assembled a feedback controller for an electro-optic modulator in order to achieve high-fidelity entangled photon pairs.'' A lot of faculty have told me that many students will just say whatever is they worked on without giving the big picture behind their work. I could just say I did electronics all day, but then how am I different from an electrical engineer? Show you understand why you did the work you did. In essence: briefly state what the goal of the experiment/collaboration is. 

	You'll also see there is a ``funnel'' at play here. I give the big picture of the collaboration at Fermilab, discuss what Caltech is doing, then I finally state what it is I do. I recommend you use the funnel in whatever it is you write. It looks simple, but we forget to use it, and it's a great way to help structure your paragraphs.

	\item ``I found that my attempts to implement traditional PID feedback control failed. However, by programming our own nonlinear algorithm, we achieved stable feedback.'' Very concise description of the problem I faced and our idea to do something entirely different. If you have multiple research experiences, you'll probably have to describe multiple difficulties, so don't bear out every detail.

	\item ``Our \$50 controller performed comparably to a \$1000 commercial controller. We inferred a fidelity of 0.9938, which I reported in a first-author publication [1].'' I stated the results of our work. Pinning numbers to what you do, if you can, grounds your sentences and makes what you write sound real. If you don't have numbers, it's no matter. But if you do, use them. I also mentioned a publication we wrote. If you have one, mention it --- remember the M.O.

	\item ``Toward the end of the summer, I used single photon detectors to show that we had realized entangled pairs of photons. During my summer at Caltech, I learned how to apply quantum information science to questions in fundamental physics.'' I wrote in the beginning of the statement that I was interested in interdisciplinary approaches to fundamental physics, so I describe how that experience plays a part. We did work in quantum information with the eventual application to fundamental physics, so I tie this paragraph back to my introduction with this sentence.
	\end{enumerate}

	\item Describe all your research experiences. This will bring you to the last paragraph, where you will describe why the specific program you're applying to is a good fit. This is where you need to mention the faculty members at that school and their research projects. Then, re-iterate how your past experiences would help you be a productive member of their group! 

	\item Being concise is vital. This is definitely a lot to discuss in two pages. Your first draft will not be concise. 90\% of writing is in the editing. I'll quote Kurt Vonnegut: ``Have the guts to cut.''

	\item If you have any publications, mention them and cite them to give the adcom an easy way to access them.

	\item If you can, prepare it in \LaTeX, it's my belief that it looks a bit more professional and gives the document an air of finality and authority.

	\item Reuse your statement, and just change the last paragraph for each school. 
\end{enumerate}

\section{CV}
\subsection{Comments}
The curriculum vit\ae\ (CV) is another very important document for your application. Compared to a r\'esum\'e, the CV has no page limit and lists everything: education, experience, awards, presentations, etc. I will discuss how to write a strong CV and what sections to include.

\subsection{Advice}
\begin{enumerate}
\item Start out with a nice template. One of my favorites is the Rezi BETA template, which I've uploaded to the Drive folder I linked to at the beginning. I recommend \LaTeX if you can though (especially with Awesome-CV), since it's very flexible.

\item The sections I think you should include (in this order) are:
\begin{enumerate}
\item Header with name, email, phone
\item Education
\item Research Experience
\item Publications 
\item Presentations and Posters 
\item Awards \& Honors 
\item Work Experience 
\item Outreach
\item Professional Development
\item Skills
\end{enumerate}
Of course, some may be irrelevant to you, so just exclude them if so. Or, you feel another heading may be better. That's fine too!

\item All you have to do is fill it out. Of course, this is the meat of the CV. Under each item, put bullet point(s) describing what that item is.

\item The hardest part of the CV is describing your work and doing so in a very active way. It's not enough to say ``I was a research assistant and researched yada-yada.'' You should address what the central question of your work is, and use very concrete verbs to describe what it is you did.

I'll give you a bad example and a good one, from my own CV. Let's start with a bad one.
\begin{quote}
$\bullet$\quad Made a feedback controller for a quantum computing experiment
\end{quote}

This example is bad for two reasons. First, ``made'' is a generic verb. It's not specific and doesn't give an idea of what you did. Second, reading this, I wonder exactly how the feedback controller works in the setup. What is it doing? We don't know.

Here's that same sentence written to be better.

\begin{quote}
$\bullet$\quad  Designed and assembled an electro-optic modulator feedback controller for producing high-fidelity time-bin entangled photons
\end{quote}

This a lot better. We've replaced ``made'' with ``designed'' and ``assembled,'' which are much more concrete. They ground this statement in reality. The rest of the example is much more specific and says exactly what this feedback controller is for.

Be conscious of how you describe each item on your CV, particularly with the verbs, which should be punchy.

\item There's no page limit, so make it as long as you like, within reason.

\item Date the items on your CV so you give people an idea of when things happened. Order things in reverse chronological order i.e., most recent items first.
\end{enumerate} 


\section{Interviews}
\subsection{Comments}
After you submit your application, the school or a PI from the school might reach out to you regarding an interview, typically online.\footnote{From what I understand, programs in the biological sciences typically have in-person interviews. I don't have experience with these, but this advice is still applicable.} The norm depends on your field. In physics, you can be admitted with or without an interview. In other fields, it's unheard of to be admitted without first being interviewed. Regardless, you want to do your best. Let's dive in.

\subsection{Advice}
\begin{enumerate}
\item The best way to approach an interview is to see it as a conversation between you and the faculty. So, it's two-way: it is a chance for the faculty to learn more about you and your interests of course, but it's a chance for you to learn more about the faculty and the program, too.

\item I recommend preparing by doing a few things. First, make sure you have a short elevator pitch. A lot of interviewers will start by asking you to describe yourself and your interests. So, practice a pitch you can give that covers who you are (major, year), previous work you've done, and your field/questions of interest.

\item Second, review your previous research experiences. Some interviewers may ask you some detailed questions about your previous work, and you'll want to make sure you can answer these questions. They won't grill you, but just make sure you're comfortable discussing the aim of your previous work, why you took on that work, and difficulties you overcame.

\item Review common questions that they could ask you, and come up with some of your own questions to ask. There is a phenomenally comprehensive list on Reddit, so I'll link it here \href{https://www.reddit.com/r/gradadmissions/comments/a4vpc4/master_list_of_interview_questions/}{(click here!)}. Review the questions you could be asked, and pick a few questions that you want the answer to, and ask them at the end of the interview.
\end{enumerate}

\section{Visiting Schools and Meeting Faculty}
\subsection{Flyins}
One very important of the graduate admissions process is talking to faculty. You can call and email professors, which obviously is great, but nothing beats face-to-face conversation, right? So, a lot of schools have started some great programs intended for URM students (most of them are for URMs, but not all, so keep reading even if you're not URM) that will fly students out to the school for a few days during the academic year. Obviously, they're taking classes then, so the schools will offer a very condensed itinerary where you'll get to: network with grad students, have face-to-face chats with faculty members you're interested in, attend workshops on drafting a successful graduate applications, and so on. Everything is paid for too.

I believe these programs can be very helpful if you plan to apply to the school, since it's less likely the professors will forget you if you've met them. Plus, once you apply to the school itself, the application will actually ask if you've attended the open house. I believe this is a slight boost to your application.

Caveat: As with everything, it's not a guarantee. I did two visits at two schools. I got rejected by one and accepted by the other. That being said, there is absolutely no way these opportunities can hurt your application; they can only help.

If you want to apply, there is a fantastic and up-to-date list I'll refer you to: \href{https://www.cientificolatino.com/diversity-preview-events}{(click here!)}.

\subsection{Seminars}
If your school has an active seminar series where they invite faculty from other schools to visit, you should check out who's on the list. 

I can't count how many times I've found somebody whose work is interesting and then I see they visited my school just a few months prior. Aw shucks, I could have talked them face-to-face. Maybe this will help you, maybe it won't. But when you have your list of faculty, just check out the seminar list.

\section{Mentorship}
Having a mentor through the process helps. I can recommend several mentorship resources.
\subsection{Resources}
\begin{enumerate}
\item Your current research advisor is always a go-to. They're invested in seeing you succeed.

\item Any friends in grad school are good, since they know that the process can be stressful, and they can be good pointers.

\item When I was going through the process, I signed up for the Cientifico Latino mentorship process, which was amazing. They pair you with a mentor (who is usually a faculty member), and you work with them for a few months and you go back and forth on drafts. They keep you accountable. Here's the link: \href{https://www.cientificolatino.com/gsmi}{(click here!)}.

\item I'm always here if you have questions or just want some advice on something the guide didn't answer!
\end{enumerate}
\newpage
\section{Timeline}
Here is a timeline of the process. This is just my recommendation; I recommend doing things early so you have ample time to polish your application. Drafting and writing is easiest in the summer.

\begin{table}[!h]
\caption{Graduate Admissions Timeline}
\centering
\begin{minipage}[t]{.7\linewidth}
\color{gray}
\rule{\linewidth}{1pt}
\ytl{Apr./May}{Secure letters of recommendation}
\ytl{May/Jun.}{Make a list of programs and faculty}
\ytl{Jul./Aug.}{Start drafting SOP and fellowship essays}
\ytl{Jul.---Sept.}{Talk to faculty}
\ytl{Sept./Oct.}{Apps open; start them and fill out biographical information}
\ytl{Oct./Nov.}{Make final edits to essays with advisors and recommenders}
\ytl{Oct.---Dec.}{Submit fellowship applications}
\ytl{Dec.---Jan.}{Submit grad school applications}
\ytl{Jan.---Feb.}{Interviews}
\bigskip
\rule{\linewidth}{1pt}%
\end{minipage}%
\end{table}

\section{Other Guides}
This is just my perspective --- there are plenty of other guides online which I think are also helpful and I'll link to them here.
\begin{enumerate}
\item \href{http://www.pgbovine.net/PhD-application-tips.htm}{A Five-Minute Guide to Ph.D. Program Applications by Prof. Guo (click here)}

\item \href{http://www.pgbovine.net/grad-school-app-tips.htm}{Advice for Ph.D. Program Applications by Prof. Guo (click here)
}

\item \href{http://www.pgbovine.net/fellowship-tips.htm}{Advice for Science Fellowship Applications: NSF, NDSEG, Hertz
 by Prof. Guo (click here)}

 \item \href{http://writeivy.com/structure-is-magic-a-guide-to-the-graduate-sop/}{Structure is Magic: A Guide to the Graduate SOP (click here)}
\end{enumerate}
\end{document}
